%================================================================%
%======  Modelo de Monografia ( UFOP - DECOM) ===================%
% Proposta de texto em conformidade com normas da ABNT ----------%
% implementadas pelo projeto abntex2, que pode ser acessado pela %
% página  http://abntex2.googlecode.com/  -----------------------%
%================================================================%
\documentclass[12pt, % tamanho da fonte
   %openright,	     % capítulos começam em página ímpar
	oneside,		  % twoside para impressão em frente e verso.  
	a4paper,			% tamanho do papel. 
	english,			% Idioma adicional para hifenização
    brazil,				% Idioma principal 
    sumario=tradicional % Comente para o sumario ser conforme a opção padrão recomendada pela ABNT NBR 6027:2012.
	]{abntex2}
	
\input{structure} 

% -- Informações para Capa e Folha de Rosto: ---------------
\titulo{XXX} 
\subtitulo{XXX}
\autor{XX} \autorcite{Aluno, Nome}
\local{Ouro Preto} \uf{MG}
\data{XX de mês de Ano} \ano{20XX}
\orientador{Marco Antonio Moreira de Carvalho}
\ttorientador{Universidade Federal de Ouro Preto}
\ttcoorientador{Universidade Federal de Ouro Preto}
\instituicao{Universidade Federal de Ouro Preto} \sigla{UFOP}
\instituto{Instituto de Ciências Exatas e Biológicas}
\departamento{Departamento de Computação}
\curso{Ciência da Computação}	
\tipotrabalho{Monografia} % Monografia (Monografia II)
\grau{Bacharel em Ciência da Computação}

%------Nomes dos membros da banca.  
\examinadorum{Prof. Dr. XXX}
\ttexaminadorum{Universidade Federal de Ouro Preto - UFOP}
\examinadordois{Prof. Dr. XXX}
\ttexaminadordois{XXX - UFXX}

% ------------------------------------------------------
\makeindex   

\begin{document} % Início do documento

\frenchspacing  % Retira espaço obsoleto entre as frases.

% ----------------------------------------------------------
% -- Elementos Pré-Textuais: -------------------------------
\pagenumbering{roman}

\imprimircapa  % Capa
\imprimirfolhaderosto % Folha de rosto
% \include{PreTextuais/FichaCatalografica}
% \include{PreTextuais/Errata}
% \include{PreTextuais/FichaAprovacao} 
% \include{PreTextuais/Dedicatoria}
% \include{PreTextuais/Agradecimento}
% \include{PreTextuais/Epigrafe}
%--------------------------------------------------------------------------
%--------------------- Resumo em Português --------------------------------
%--------------------------------------------------------------------------

\setlength{\absparsep}{18pt} % ajusta o espaçamento dos parágrafos do resumo
\begin{resumo}


 \vspace{\onelineskip}
 \noindent
 \textbf{Palavras-chave}: Manufatura Flexível. XXX. Revenimento Paralelo. 

\end{resumo}

%--------------------------------------------------------------------------
%--------------------- Resumo em Inglês --------------------------------
%--------------------------------------------------------------------------
\begin{resumo}[Abstract]
 \begin{otherlanguage*}{english}
  
  \vspace{\onelineskip}
   \noindent 
   \textbf{Keywords}: Flexible Manufacturing. XXX. Parallel Tempering.
 \end{otherlanguage*}
\end{resumo} % (Abstract no mesmo arquivo)

% As listas abaixo são opcionais. Caso o trabalho não possua alguma(s) dela(s) basta comentar os seus respectivos comandos.

% Lista de Figuras. 
\pdfbookmark[0]{\listfigurename}{lof}
\listoffigures*   
\cleardoublepage
% lista de Tabelas
\pdfbookmark[0]{\listtablename}{lot}
\listoftables*  
\cleardoublepage
% Lista de Algoritmos
\pdfbookmark[0]{\listalgorithmcfname}{lof}
\listofalgorithmes   
%\cleardoublepage

% Lista de Siglas e Símbolos. Estas listas são criadas manualmente e seus arquivos estão na pasta PreTextuais.
% ---------------------------------------------------
% ------ Lista de abreviaturas e siglas -------------
% ---------------------------------------------------
\begin{siglas}
    \item [FMS] flexible manufacturing system
    \item [GPCA] Greedy Pipe Construction Algorithm
    \item [HMLV] high-mix, low-volume
    \item [KTNS] Keep Tool Needed Soonest
    \item [MCMC] Monte Carlo Markov Chain
    \item [PT] \textit{parallel tempering}
    \item [SSP] job sequencing and tool switching problem
\end{siglas}
% % ---------------------------------------------------
% ----------- Lista de símbolos ---------------------
% ---------------------------------------------------

\begin{simbolos}
    % \item [TESTE] Teste
\end{simbolos}

% Sumário:
\pdfbookmark[0]{\contentsname}{toc}
\tableofcontents*
\cleardoublepage

%% ------------- Capítulos ----------------------%%
\pagenumbering{arabic} \setcounter{page}{1}
\textual 
\chapter{Introdução} \label{Introducao}

\section{Justificativa}

\section{Objetivos}

\section{Organização do Trabalho}

\chapter{Trabalhos relacionados} \label{RevisaoBibliografica}

\chapter{Fundamentação Teórica} \label{fundamentacao}

\section{O problema XXX}


\section{RO método XXX}


\chapter{Desenvolvimento} \label{desenvolvimento}


\section{Análise dos dados}

\section{Novas instâncias} 

\section{Função de avaliação}

\section{Solução inicial}

\section{Codificação e decodificação}

\section{Estruturas de vizinhança}

\chapter{Experimentos Computacionais} \label{experimentos}

\section{Experimentos preliminares}

\section{Comparação com o estado da arte}

\chapter{Plano de Atividades Restantes} \label{plano}


\begin{table}[H]
\label{table:planoAtividades}
\caption{Planejamento de atividades para Monografia II.}
\begin{tabular}{@{}lcccc@{}}
\toprule
Atividades                                             & \multicolumn{1}{l}{Mês 1} & \multicolumn{1}{l}{Mês 2} & \multicolumn{1}{l}{Mês 3} & \multicolumn{1}{l}{Mês 4} \\ \midrule
Adaptação da API do PT para a versão específica do SSP & X                         & X                         &                           &                           \\
Implementação do modelo matemático &                   & X                         &                           &                           \\
Realização de experimentos computacionais              &                           & X                         & X                         &                           \\
Testes para calibração de parâmetros                   &                           & X                         & X                         &                           \\
Descrição dos experimentos                             &                           &                           & X                         &                           \\
Análise dos experimentos                               &                           &                           & X                         & X                         \\
Conclusão da Monografia                                &                           &                           &                           & X                         \\ \bottomrule
\end{tabular}
\end{table}


\include{Capitulos/Cap7-Conclusão}


%% -------------- Elementos Pós-Textuais -----------------%%
\postextual  
\bibliography{bibliografia} % Referências bibliográficas
% \include{PosTextuais/Apendice}
% \include{PosTextuais/Anexo}
\phantompart  \printindex  % Índice Remissivo
% ----------------------------------------------------------
\end{document}  % fim do documento
