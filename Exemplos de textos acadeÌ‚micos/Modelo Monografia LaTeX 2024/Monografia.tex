%================================================================%
%======  Modelo de Monografia ( UFOP - DECOM) ===================%
% Proposta de texto em conformidade com normas da ABNT ----------%
% implementadas pelo projeto abntex2, que pode ser acessado pela %
% página  http://abntex2.googlecode.com/  -----------------------%
%================================================================%
\documentclass[12pt, % tamanho da fonte
   %openright,	     % capítulos começam em página ímpar
	oneside,		  % twoside para impressão em frente e verso.  
	a4paper,			% tamanho do papel. 
	english,			% Idioma adicional para hifenização
    brazil,				% Idioma principal 
    sumario=tradicional % Comente para o sumario ser conforme a opção padrão recomendada pela ABNT NBR 6027:2012.
	]{abntex2}
	
%------------------------------------------------------------
%------------    MEUS IMPORTS   -----------------------      
\usepackage{float} 
\usepackage{makecell}
\usepackage{array,multirow,graphicx}

\newcommand{\STAB}[1]{\begin{tabular}{@{}c@{}}#1\end{tabular}}

%------------------------------------------------------------
%------------    Estrutura do texto   -----------------------      

% Pacotes Básicos:
\usepackage[T1]{fontenc}		% Seleção de códigos de fonte.
\usepackage[utf8]{inputenc}		% Codificação do documento (conversão automática dos acentos)
\usepackage{lastpage}		    % Usado pela Ficha catalográfica
\usepackage{indentfirst}		
\usepackage{color}				% Controle das cores
\usepackage{graphicx}			% Inclusão de gráficos
\usepackage{tabularx}
\usepackage{microtype} 		  	% Melhorias da justificação
\usepackage{pdfpages}           %inserir páginas em PDF
% Pacotes Extras:
\usepackage{amsmath,amsthm}     % Símbolos Matemáticos
\usepackage{threeparttable}
\usepackage[portuguese, ruled, linesnumbered,commentsnumbered, algo2e, vlined, lined, boxed, algochapter]{algorithm2e} % Algoritmos 
\usepackage{hyperref}      % Criação de links.

% Escolha da formatação das referências Bibliográficas: 
\usepackage[alf,abnt-etal-list=0,abnt-etal-cite=3]{abntex2cite}	% Citações padrão ABNT  (AUTOR, ANO)
\usepackage{etoolbox}
\usepackage{placeins}
%\usepackage[num]{abntex2cite}  % Citações numéricas (1)
%\citebrackets[] % Usar este comando para a citação numérica aparecer com [].

% Numeração das Figuras e Tabelas
\counterwithin{figure}{chapter}
\counterwithin{table}{chapter}

% Defininção de Cores:
\definecolor{blue}{RGB}{25,25,112}
\makeatletter % informações do PDF
\hypersetup{
    %pagebackref= false,
	pdftitle={\@title}, 
	pdfauthor={\@author},
    %pdfsubject={\imprimirpreambulo},
	pdfcreator={LaTeX with abnTeX2},
	pdfkeywords={abnt}{latex}{abntex}{abntex2}{trabalho acadêmico}, 
	colorlinks=true,    % false: boxed links; true: colored links
    linkcolor=blue,     % color of internal links
    citecolor=blue,     % color of links to bibliography
    filecolor=magenta, 	% color of file links
	urlcolor=blue,
	bookmarksdepth=4
}
\makeatother

% -------------------------------------------- 
% Espaçamentos entre linhas e parágrafos 
\setlength{\parindent}{1.3cm} % O tamanho do parágrafo

% Controle do espaçamento entre um parágrafo e outro:
\setlength{\parskip}{0.2cm}  % tente também \onelineskip

% Definição de ambientes matemáticos em português 
\newtheorem{teorema}{Teorema}[chapter]
\newtheorem{axioma}{Axioma}[chapter]
\newtheorem{corolario}{Corolário}[chapter]
\newtheorem{lema}{Lema}[chapter]
\newtheorem{proposicao}{Proposição}[chapter]
\newtheorem{definicao}{Definição}[chapter]
\newtheorem{exemplo}{Exemplo}[chapter]
\newtheorem{observacao}{Observação}[chapter]

% Novos Comandos
\usepackage{tgtermes}
\renewcommand{\ABNTEXchapterfont}{\rmfamily\bfseries}

% Variáveis adicionais
\providecommand{\imprimirautorcite}{}
\newcommand{\autorcite}[1]{\renewcommand{\imprimirautorcite}{#1}} 
\providecommand{\imprimirsubtitulo}{}
\newcommand{\subtitulo}[1]{\renewcommand{\imprimirsubtitulo}{#1}} 
\providecommand{\imprimirsigla}{}
\newcommand{\sigla}[1]{\renewcommand{\imprimirsigla}{#1}}
\providecommand{\imprimiruf}{}
\newcommand{\uf}[1]{\renewcommand{\imprimiruf}{#1}}
\providecommand{\imprimircurso}{}
\newcommand{\curso}[1]{\renewcommand{\imprimircurso}{#1}}
\providecommand{\imprimirinstituto}{}
\newcommand{\instituto}[1]{\renewcommand{\imprimirinstituto}{#1}}
\providecommand{\imprimirdepartamento}{}
\newcommand{\departamento}[1]{\renewcommand{\imprimirdepartamento}{#1}}
\providecommand{\imprimirano}{}
\newcommand{\ano}[1]{\renewcommand{\imprimirano}{#1}}
\providecommand{\imprimirgrau}{}
\newcommand{\grau}[1]{\renewcommand{\imprimirgrau}{#1}}
\providecommand{\imprimirexaminadorum}{}
\newcommand{\examinadorum}[1]{
    \renewcommand{\imprimirexaminadorum}{#1}}
\providecommand{\imprimirexaminadordois}{}
\newcommand{\examinadordois}[1]{
    \renewcommand{\imprimirexaminadordois}{#1}}
\providecommand{\imprimirexaminadortres}{}
\newcommand{\examinadortres}[1]{
    \renewcommand{\imprimirexaminadortres}{#1}}
\providecommand{\imprimirexaminadorquatro}{}
\newcommand{\examinadorquatro}[1]{
    \renewcommand{\imprimirexaminadorquatro}{#1}}
\providecommand{\imprimirttorientador}{}
\newcommand{\ttorientador}[1]{
    \renewcommand{\imprimirttorientador}{#1}} 
\providecommand{\imprimirttcoorientador}{}
\newcommand{\ttcoorientador}[1]{
    \renewcommand{\imprimirttcoorientador}{#1}}
\providecommand{\imprimirttexaminadorum}{}
\newcommand{\ttexaminadorum}[1]{
    \renewcommand{\imprimirttexaminadorum}{#1}}
\providecommand{\imprimirttexaminadordois}{}
\newcommand{\ttexaminadordois}[1]{\renewcommand{
        \imprimirttexaminadordois}{#1}}
\providecommand{\imprimirttexaminadortres}{}
\newcommand{\ttexaminadortres}[1]{
    \renewcommand{\imprimirttexaminadortres}{#1}}
\providecommand{\imprimirttexaminadorquatro}{}
\newcommand{\ttexaminadorquatro}[1]{
    \renewcommand{\imprimirttexaminadorquatro}{#1}}
		
%----------------------------------------------------
\renewcommand{\imprimircapa}{ % Capa 
\begin{capa}
        \begin{center}
                \begin{DoubleSpace}
                \MakeUppercase{\imprimirinstituicao } \\
                 \MakeUppercase{\imprimirinstituto } \\
                \MakeUppercase{\imprimirdepartamento} \\
                \end{DoubleSpace}
                \vspace{5cm}
				\MakeUppercase{\imprimirautor}  \\
                \imprimirorientadorRotulo ~\imprimirorientador \\
                \imprimircoorientadorRotulo ~\imprimircoorientador \\
                        				
				\vspace{5cm}
             \textbf{{\large\MakeUppercase{\imprimirtitulo}}} \\
			 \textbf{{\large \MakeUppercase{\imprimirsubtitulo}}} \\
				\vfill
        {\large{\imprimirlocal, ~\imprimiruf \\ \imprimirano  }}
        \end{center}
\end{capa}   
} % Capa



%----------------------------------------------------
\renewcommand{\imprimirfolhaderosto}{% folha de rosto
       \begin{center}
                \MakeUppercase{\imprimirinstituicao } \\
                 \MakeUppercase{\imprimirinstituto } \\
                \MakeUppercase{\imprimirdepartamento} \\
                
                \vspace{4cm}
				\MakeUppercase{\imprimirautor}  \\
				\vspace{2cm}
			    \begin{DoubleSpace}
                \MakeUppercase{\textbf{\imprimirtitulo} } \\
                \MakeUppercase{\textbf{\imprimirsubtitulo}} \\
                \end{DoubleSpace} 
      \end{center}
    \vfill 
    \begin{flushright} 
    \parbox{0.6\linewidth}{
		\imprimirtipotrabalho~ apresentada ao Curso de \imprimircurso~ da \imprimirinstituicao~ como parte dos
		requisitos necessários para a obtenção do grau de \imprimirgrau. \\ \\
		\textbf{\imprimirorientadorRotulo}~\imprimirorientador \\
		\textbf{\imprimircoorientadorRotulo}~\imprimircoorientador}
   \end{flushright} 
		\vfill
   \begin{center}
   {\large{\imprimirlocal, ~ \imprimiruf \\
   \imprimirano} }
   \end{center} }  % folha de rosto

%----------------------------------------------------


 

% -- Informações para Capa e Folha de Rosto: ---------------
\titulo{XXX} 
\subtitulo{XXX}
\autor{XX} \autorcite{Aluno, Nome}
\local{Ouro Preto} \uf{MG}
\data{XX de mês de Ano} \ano{20XX}
\orientador{Marco Antonio Moreira de Carvalho}
\ttorientador{Universidade Federal de Ouro Preto}
\ttcoorientador{Universidade Federal de Ouro Preto}
\instituicao{Universidade Federal de Ouro Preto} \sigla{UFOP}
\instituto{Instituto de Ciências Exatas e Biológicas}
\departamento{Departamento de Computação}
\curso{Ciência da Computação}	
\tipotrabalho{Monografia} % Monografia (Monografia II)
\grau{Bacharel em Ciência da Computação}

%------Nomes dos membros da banca.  
\examinadorum{Prof. Dr. XXX}
\ttexaminadorum{Universidade Federal de Ouro Preto - UFOP}
\examinadordois{Prof. Dr. XXX}
\ttexaminadordois{XXX - UFXX}

% ------------------------------------------------------
\makeindex   

\begin{document} % Início do documento

\frenchspacing  % Retira espaço obsoleto entre as frases.

% ----------------------------------------------------------
% -- Elementos Pré-Textuais: -------------------------------
\pagenumbering{roman}

\imprimircapa  % Capa
\imprimirfolhaderosto % Folha de rosto
% % ---------------------------------------------------------------
% ----------------  Ficha Catalográfica  -------------------------
% ---------------------------------------------------------------
% Modelo de ficha catalográfica. Você deverá substituir esta
% folha na versão final da monografia por um pdf fornecido pela 
% biblioteca. Salve o modelo oficial como ficha_catalografica.pdf
% e use o comando abaixo para inseri-lo na versão final do texto.

%\begin{fichacatalografica}
%    \includepdf{ficha_catalografica.pdf}
%\end{fichacatalografica}



%% Modelo de Como fazer a Ficha Catalográfica:

\begin{fichacatalografica}
	\sffamily
	\vspace*{\fill}					% Posição vertical
	\begin{center}					% Minipage Centralizado
	\fbox{\begin{minipage}[c][8cm]{14cm}		% Largura
	\small
	\imprimirautorcite.
	%Sobrenome, Nome do autor
	
	\hspace{0.5cm}  \\
	\imprimirtitulo  / \imprimirautor. --, \imprimirano-
	
	\hspace{0.5cm} \pageref{LastPage} p. 1 :il. (colors; grafs; tabs).\\
	
	\hspace{0.5cm} \imprimirorientadorRotulo~\imprimirorientador\\
	
	\hspace{0.5cm}
	\parbox[t]{\textwidth}{\imprimirtipotrabalho~--~\imprimirinstituicao, ~ \\
	\imprimirinstituto, ~\imprimirdepartamento,~\imprimirano.}\\
	
	\hspace{0.5cm} % Palavras-chave do trabalho
		1. Palavra-chave 1.
		2. Palavra-chave 2.
		2. Palavra-chave 3.
		3. Palavra-chave 4.
		4. Palavra-chave 5.
		I. \imprimirorientador.
		II. \imprimirinstituicao.
		III. \imprimirtitulo  			
	\end{minipage} }
	\end{center}
\end{fichacatalografica}



% \begin{errata}

\noindent AUTOR, \textbf{\imprimirtitulo} \imprimirsubtitulo. nº de páginas. \imprimirtipotrabalho - 
\imprimirdepartamento, \imprimirinstituicao, \imprimirlocal, \imprimirano.

\begin{table}[!ht]
\centering
\begin{tabular}{|c|c|c|c|} \hline
\textbf{Página} & \textbf{Linha} & \textbf{Onde se lê} & \textbf{Leia-se} \\ \hline
16 & 10 & &  \\ \hline
\end{tabular}
\end{table}


Este elemento pré-textual é opcional e deve ser inserido no texto, geralmente após o trabalho já ter sido impresso, após a folha de rosto. Deve conter a referência do trabalho e o texto da errata com a indicação da página a linha \cite{NBR14724:2011}.

\end{errata}
% % ---------------------------------------------------------------
% ----------------  Folha de aprovação  -------------------------
% ---------------------------------------------------------------
% Modelo de Folha de aprovação. Você deverá substituir esta folha na versão final da monografia por um pdf fornecido pelo colegiado do seu curso. Salve o modelo oficial como 
% folhadeaprovacao_final.pdf e use o comando abaixo para inseri-lo na versão final do texto. 
% A versão abaixo foi feita seguindo as normas ABNT NBR 14724:2011 em vigor.

%\begin{fichacatalografica}
%    \includepdf{folhadeaprovacao_final.pdf}
%\end{fichacatalografica} Esta folha será 


\begin{folhadeaprovacao}


\begin{center}
     {\large \imprimirautor}\\
       	\vspace{2cm}	
    \begin{DoubleSpace}    
    {\large \textbf{\MakeUppercase{\imprimirtitulo}}} \\
    {\large \textbf{\MakeUppercase{\imprimirsubtitulo}}}
    \end{DoubleSpace}
		\vspace{1cm}
        
\end{center}		


\begin{flushright} 
    \parbox{0.6\linewidth}{
		\imprimirtipotrabalho~ apresentada ao Curso de \imprimircurso~ da \imprimirinstituicao~ como parte dos
		requisitos necessários para a obtenção do grau em \imprimirgrau. \\}
   \end{flushright} 

\noindent Aprovada em \imprimirlocal,~ \imprimirdata. 
\begin{center}
\vfill
       \rule{10cm}{.1pt} \\
       {\imprimirorientador} \\ {\imprimirttorientador} \\
			 Orientador 
       \vfill
			 \ifdefvoid{\imprimircoorientador}{}{
       \rule{10cm}{.1pt} \\
       \imprimircoorientador \\ \imprimirttcoorientador \\ Coorientador }
			 \vfill
       \rule{10cm}{.1pt} \\
       {\imprimirexaminadorum} \\ {\imprimirttexaminadorum} \\ Examinador
        \vfill
        \ifdefvoid{\imprimirexaminadordois}{}{
        \rule{10cm}{.1pt} \\
        \imprimirexaminadordois \\ \imprimirttexaminadordois \\ Examinador}
				\vfill
        \ifdefvoid{\imprimirexaminadortres}{}{
        \rule{10cm}{.1pt} \\
        \imprimirexaminadortres \\ \imprimirttexaminadortres \\ Examinador}
				\vfill
        \ifdefvoid{\imprimirexaminadorquatro}{}{
        \rule{10cm}{.1pt} \\
        \imprimirexaminadorquatro \\ \imprimirttexaminadorquatro \\ Examinador}
\end{center}
  
\end{folhadeaprovacao}
% --- 
% \begin{dedicatoria}
   \vspace*{\fill}
   \centering
   \noindent
   \textit{ Espaço para prestar uma homenagem ou dedicar o trabalho a alguém.} 
	 \vspace*{\fill}
\end{dedicatoria}
% \begin{agradecimentos}

Espaço para agradecer às pessoas e/ou instituições que contribuíram de forma relevante à elaboração do trabalho.

\end{agradecimentos}

% \begin{epigrafe}
    \vspace*{\fill}
    
	\begin{flushright}
   Alguma citação importante para a construção do trabalho. A fonte deve ser indicada e também deve constar na lista de referências bibliográficas.
	\end{flushright}
    
\end{epigrafe}
%--------------------------------------------------------------------------
%--------------------- Resumo em Português --------------------------------
%--------------------------------------------------------------------------

\setlength{\absparsep}{18pt} % ajusta o espaçamento dos parágrafos do resumo
\begin{resumo}


 \vspace{\onelineskip}
 \noindent
 \textbf{Palavras-chave}: Manufatura Flexível. XXX. Revenimento Paralelo. 

\end{resumo}

%--------------------------------------------------------------------------
%--------------------- Resumo em Inglês --------------------------------
%--------------------------------------------------------------------------
\begin{resumo}[Abstract]
 \begin{otherlanguage*}{english}
  
  \vspace{\onelineskip}
   \noindent 
   \textbf{Keywords}: Flexible Manufacturing. XXX. Parallel Tempering.
 \end{otherlanguage*}
\end{resumo} % (Abstract no mesmo arquivo)

% As listas abaixo são opcionais. Caso o trabalho não possua alguma(s) dela(s) basta comentar os seus respectivos comandos.

% Lista de Figuras. 
\pdfbookmark[0]{\listfigurename}{lof}
\listoffigures*   
\cleardoublepage
% lista de Tabelas
\pdfbookmark[0]{\listtablename}{lot}
\listoftables*  
\cleardoublepage
% Lista de Algoritmos
\pdfbookmark[0]{\listalgorithmcfname}{lof}
\listofalgorithmes   
%\cleardoublepage

% Lista de Siglas e Símbolos. Estas listas são criadas manualmente e seus arquivos estão na pasta PreTextuais.
% ---------------------------------------------------
% ------ Lista de abreviaturas e siglas -------------
% ---------------------------------------------------
\begin{siglas}
    \item [FMS] flexible manufacturing system
    \item [GPCA] Greedy Pipe Construction Algorithm
    \item [HMLV] high-mix, low-volume
    \item [KTNS] Keep Tool Needed Soonest
    \item [MCMC] Monte Carlo Markov Chain
    \item [PT] \textit{parallel tempering}
    \item [SSP] job sequencing and tool switching problem
\end{siglas}
% % ---------------------------------------------------
% ----------- Lista de símbolos ---------------------
% ---------------------------------------------------

\begin{simbolos}
    % \item [TESTE] Teste
\end{simbolos}

% Sumário:
\pdfbookmark[0]{\contentsname}{toc}
\tableofcontents*
\cleardoublepage

%% ------------- Capítulos ----------------------%%
\pagenumbering{arabic} \setcounter{page}{1}
\textual 
\chapter{Introdução} \label{Introducao}

\section{Justificativa}

\section{Objetivos}

\section{Organização do Trabalho}

\chapter{Trabalhos relacionados} \label{RevisaoBibliografica}

\chapter{Fundamentação Teórica} \label{fundamentacao}

\section{O problema XXX}


\section{RO método XXX}


\chapter{Desenvolvimento} \label{desenvolvimento}


\section{Análise dos dados}

\section{Novas instâncias} 

\section{Função de avaliação}

\section{Solução inicial}

\section{Codificação e decodificação}

\section{Estruturas de vizinhança}

\chapter{Experimentos Computacionais} \label{experimentos}

\section{Experimentos preliminares}

\section{Comparação com o estado da arte}

\chapter{Plano de Atividades Restantes} \label{plano}


\begin{table}[H]
\label{table:planoAtividades}
\caption{Planejamento de atividades para Monografia II.}
\begin{tabular}{@{}lcccc@{}}
\toprule
Atividades                                             & \multicolumn{1}{l}{Mês 1} & \multicolumn{1}{l}{Mês 2} & \multicolumn{1}{l}{Mês 3} & \multicolumn{1}{l}{Mês 4} \\ \midrule
Adaptação da API do PT para a versão específica do SSP & X                         & X                         &                           &                           \\
Implementação do modelo matemático &                   & X                         &                           &                           \\
Realização de experimentos computacionais              &                           & X                         & X                         &                           \\
Testes para calibração de parâmetros                   &                           & X                         & X                         &                           \\
Descrição dos experimentos                             &                           &                           & X                         &                           \\
Análise dos experimentos                               &                           &                           & X                         & X                         \\
Conclusão da Monografia                                &                           &                           &                           & X                         \\ \bottomrule
\end{tabular}
\end{table}


\chapter{Conclusão} \label{conclusao}


%% -------------- Elementos Pós-Textuais -----------------%%
\postextual  
\bibliography{bibliografia} % Referências bibliográficas
% %-------------------------------------------------------------
%---------------------- Apêndices ----------------------------
%-------------------------------------------------------------

\begin{apendicesenv}
\partapendices  % Indica o início dos Apendices

\chapter{Diferença entre Anexo e Apêndice}


Os apêndices ``São textos ou documentos elaborados pelo autor, a fim de complementarem sua argumentação, sem prejuízo da unidade nuclear do trabalho'' \cite{NBR14724:2011}. Podem ser incluídos nos apêndices:  os questionários da pesquisas, as tabulação de dados, ilustrações e outros documentos que necessariamente foram preparados pelo autor. Já os anexos, em conformidade com a norma \cite{NBR14724:2011}``são textos ou documentos não elaborados pelo autor, que servem de fundamentação, comprovação ou ilustração à parte do trabalho'', como por exemplo leis, ilustrações, demonstrações de formulas, tabulações de dados de trabalhos referenciados, etc.



\chapter{Formatação}

Os apêndices devem ser identificados por letras maiúsculas consecutivas (APÊNDICE A, APÊNDICE B, etc), travessão e os respectivos títulos, devendo estar
centralizados na folha. 

\end{apendicesenv}

% %----------------------------------------------------------------
%---------------------- Anexos ----------------------------------
%----------------------------------------------------------------

\begin{anexosenv}
\partanexos   % indica o início dos anexos
\chapter{Exemplo}

Utiliza a mesma formatação dos apêndices.


\end{anexosenv}

\phantompart  \printindex  % Índice Remissivo
% ----------------------------------------------------------
\end{document}  % fim do documento
